\documentclass[a4paper, 14pt]{article}
\usepackage[T1]{fontenc}
\usepackage[utf8]{inputenc}
\usepackage{hyperref}
\hypersetup{
    colorlinks=true,     
    urlcolor=black,
}
\usepackage[english,russian]{babel}
\usepackage{setspace}
\singlespacing
\usepackage{soulutf8}
\usepackage{indentfirst}
\usepackage[top=30mm]{geometry}
\rhead{}
\lhead{}
\renewcommand{\headrulewidth}{0mm}
\usepackage{tocloft}
\usepackage{listings}
\usepackage[dvipsnames]{xcolor}
\usepackage{graphicx}
\usepackage{tabularx}
\setlength{\tabcolsep}{8.5pt}
\renewcommand{\arraystretch}{2}
\usepackage{changepage}

\lstdefinestyle{mystyle}{
    language=C++,
    commentstyle=\color{YellowGreen},
    keywordstyle=\color{RedViolet},
    numberstyle=\tiny\color{Grey},
    stringstyle=\color{BurntOrange},
    basicstyle=\footnotesize,
    breakatwhitespace=false,         
    breaklines=true,                 
    captionpos=b,                    
    keepspaces=true,                 
    numbers=left,                    
    numbersep=5pt,                  
    showspaces=false,                
    showstringspaces=false,
    showtabs=false,                  
    tabsize=4,
    literate=
        {а}{{\selectfont\char224}}1
        {б}{{\selectfont\char225}}1
        {в}{{\selectfont\char226}}1
        {г}{{\selectfont\char227}}1
        {д}{{\selectfont\char228}}1
        {е}{{\selectfont\char229}}1
        {ё}{{\"e}}1
        {ж}{{\selectfont\char230}}1
        {з}{{\selectfont\char231}}1
        {и}{{\selectfont\char232}}1
        {й}{{\selectfont\char233}}1
        {к}{{\selectfont\char234}}1
        {л}{{\selectfont\char235}}1
        {м}{{\selectfont\char236}}1
        {н}{{\selectfont\char237}}1
        {о}{{\selectfont\char238}}1
        {п}{{\selectfont\char239}}1
        {р}{{\selectfont\char240}}1
        {с}{{\selectfont\char241}}1
        {т}{{\selectfont\char242}}1
        {у}{{\selectfont\char243}}1
        {ф}{{\selectfont\char244}}1
        {х}{{\selectfont\char245}}1
        {ц}{{\selectfont\char246}}1
        {ч}{{\selectfont\char247}}1
        {ш}{{\selectfont\char248}}1
        {щ}{{\selectfont\char249}}1
        {ъ}{{\selectfont\char250}}1
        {ы}{{\selectfont\char251}}1
        {ь}{{\selectfont\char252}}1
        {э}{{\selectfont\char253}}1
        {ю}{{\selectfont\char254}}1
        {я}{{\selectfont\char255}}1
        {А}{{\selectfont\char192}}1
        {Б}{{\selectfont\char193}}1
        {В}{{\selectfont\char194}}1
        {Г}{{\selectfont\char195}}1
        {Д}{{\selectfont\char196}}1
        {Е}{{\selectfont\char197}}1
        {Ё}{{\"E}}1
        {Ж}{{\selectfont\char198}}1
        {З}{{\selectfont\char199}}1
        {И}{{\selectfont\char200}}1
        {Й}{{\selectfont\char201}}1
        {К}{{\selectfont\char202}}1
        {Л}{{\selectfont\char203}}1
        {М}{{\selectfont\char204}}1
        {Н}{{\selectfont\char205}}1
        {О}{{\selectfont\char206}}1
        {П}{{\selectfont\char207}}1
        {Р}{{\selectfont\char208}}1
        {С}{{\selectfont\char209}}1
        {Т}{{\selectfont\char210}}1
        {У}{{\selectfont\char211}}1
        {Ф}{{\selectfont\char212}}1
        {Х}{{\selectfont\char213}}1
        {Ц}{{\selectfont\char214}}1
        {Ч}{{\selectfont\char215}}1
        {Ш}{{\selectfont\char216}}1
        {Щ}{{\selectfont\char217}}1
        {Ъ}{{\selectfont\char218}}1
        {Ы}{{\selectfont\char219}}1
        {Ь}{{\selectfont\char220}}1
        {Э}{{\selectfont\char221}}1
        {Ю}{{\selectfont\char222}}1
        {Я}{{\selectfont\char223}}1
}

\lstset{style=mystyle}

\begin{document}

\thispagestyle{empty}	
\begin{center}
	Московский авиационный институт
	
	(Национальный исследовательский университет)
	
	Факультет информационных технологий и прикладной математики
	
	Кафедра вычислительной математики и программирования
	
\end{center}
\vspace{40ex}
\begin{center}
	\textbf{\large{Лабораторная работа №5-7 по курсу\linebreak \textquotedblleft Операционные системы\textquotedblright}}
\end{center}
\vspace{35ex}
\begin{flushright}
	\textit{Студент: } Немкова Анастасия Романовна
	
	\vspace{2ex}
	\textit{Группа: } М8О-208Б-22
	
	\vspace{2ex}
	\textit{Преподаватель: } Миронов Евгений Сергеевич
	
	\vspace{2ex}
	\textit{Вариант: 9} 
	
	\vspace{2ex}
	\textit{Оценка: } \underline{\quad\quad\quad\quad\quad\quad}
	
	 \vspace{2ex}
	\textit{Дата: } \underline{\quad\quad\quad\quad\quad\quad}
	
	\vspace{2ex}
	\textit{Подпись: } \underline{\quad\quad\quad\quad\quad\quad}
	
\end{flushright}

\vspace{5ex}

\begin{vfill}
	\begin{center}
		Москва, 2023
	\end{center}	
\end{vfill}
\newpage

\begin{center}
\section*{Содержание}   
\end{center}
\vspace{5ex}
\begin{enumerate}
  \item Репозиторий
  \item Постановка задачи
  \item Общие сведения о программе
  \item Общий метод и алгоритм решения
  \item Исходный код
  \item Демонстрация работы программы
  \item Вывод
\end{enumerate}
\newpage


\section*{Репозиторий}   
\vspace{2ex}
\url{https://github.com/anastasia-nemkova/OS_labs}

\section*{Постановка задачи}   
\textbf{Цель работы:}
\vspace{2ex}

Приобретение практических навыков в:
\begin{itemize}
    \item Управлении серверами сообщений (№5)
    \item Применение отложенных вычислений (№6)
    \item Интеграция программных систем друг с другом (№7)
\end{itemize}

\vspace{4ex}
\textbf{Задание:}
\vspace{2ex}

Реализовать распределенную систему по асинхронной обработке запросов. В данной  распределенной системе должно существовать 2 вида узлов: «управляющий» и «вычислительный». Необходимо объединить данные узлы в соответствии с той топологией, которая определена вариантом. Связь между узлами необходимо осуществить при помощи технологии очередей сообщений. Также в данной системе необходимо предусмотреть проверку доступности узлов в соответствии с вариантом. При убийстве («kill -9») любого вычислительного узла система должна пытаться максимально сохранять свою работоспособность, а именно все дочерние узлы убитого узла могут стать недоступными, но родительские узлы должны сохранить свою работоспособность.

Управляющий узел отвечает за ввод команд от пользователя и отправку этих команд на вычислительные узлы. Список основных поддерживаемых команд:\newline

\textbf{Создание нового вычислительного узла}

Формат команды: create id [parent]

id – целочисленный идентификатор нового вычислительного узла

parent – целочисленный идентификатор родительского узла

Формат вывода:

«Ok: pid», где pid – идентификатор процесса для созданного вычислительного узла

«Error: Already exists» - вычислительный узел с таким идентификатором уже существует

«Error: Parent not found» - нет такого родительского узла с таким идентификатором

«Error: Parent is unavailable» - родительский узел существует, но по каким-то причинам с ним не удается связаться

«Error: [Custom error]» - любая другая обрабатываемая ошибка

Пример:

> create 10 5

Ok: 3128\newline

\textbf{Исполнение команды на вычислительном узле}

Формат команды: exec id [params]

id – целочисленный идентификатор вычислительного узла, на который отправляется команда

Формат вывода:

«Ok:id: [result]», где result – результат выполненной команды

«Error:id: Not found» - вычислительный узел с таким идентификатором не найден

«Error:id: Node is unavailable» - по каким-то причинам не удается связаться с вычислительным узлом

«Error:id: [Custom error]» - любая другая обрабатываемая ошибка
\newpage
\begin{figure}

\textit{Вариант 9:}

\textbf{Топология}

    \centering
    \includegraphics[width=0.5\linewidth]{Топология.png}
    \label{fig:enter-label}
\end{figure}

Все вычислительные узлы находятся в списке. Есть только один управляющий узел. Чтобы добавить новый вычислительный узел к управляющему, то необходимо выполнить команду: create id -1.\newline

\textbf{Тип команд на вычислительных узлах}

Набор команд 2 (локальный целочисленный словарь)

Формат команды сохранения значения: exec id name value

id – целочисленный идентификатор вычислительного узла, на который отправляется команда

name – ключ, по которому будет сохранено значение (строка формата [A-Za-z0-9]+)

value – целочисленное значение

Формат команды загрузки значения: exec id name

Пример:

> exec 10 MyVar

Ok:10: 'MyVar' not found

> exec 10 MyVar 5

Ok:10

> exec 12 MyVar

Ok:12: 'MyVar' not found

> exec 10 MyVar

Ok:10: 5

> exec 10 MyVar 7

Ok:10

> exec 10 MyVar

Ok:10: 7
\newpage

\textbf{Тип проверки доступности узлов}

Команда проверки 2

Формат команды: ping id

Команда проверяет доступность конкретного узла. Если узла нет, то необходимо выводить ошибку: «Error: Not found» 

Пример:

> ping 10

Ok: 1 // узел 10 доступен

> ping 17

Ok: 0 // узел 17 недоступен

\section*{Общие сведения о программе}
Программа компилируется из файлов socket.cpp с функциями для работы с сокетами, topology.cpp с топологией списка, control\texttt{\_}node.cpp с управляющим узлом, calcutation\texttt{\_}node.cpp с вычислительным узлом. Также имеются заголовочные файлы socket.hpp и topology.hpp и тестовый файл lab5-7\texttt{\_}test.cpp. В программе работы были использованы следующие системные вызовы:

\begin{itemize}
    \item fork() - создание нового процесса
    \item execl() - замена текустщего процесса на новый
    \item bind() - привязка сокета к локальному адресу и порту
    \item coonect() - установка соединения сокета с удаленным адресом и портом
    \item unbind() - отключение сокета от локального адреса и порта
    \item disconnect() - разрыв соединения с удаленным сокетом
    \item send() - отправка сообщений через сокет
    \item revc() - приём сообщений через сокет
    \item set() - установка опций сокета
\end{itemize}

\section*{Общий метод и алгоритм решения}

Управляющий узел отвечает за создание, управление и завершение работы вычислительных узлов. При создании нового узла происходит создание нового процесса, к которому подключается новый сокет для обмена сообщениями с родительским узлом. Для коммуникации между узлами используются ZeroMQ сокеты, которые позволяют отправлять и принимать сообщения через TCP протокол.

Каждый вычислительный узел имеет свой идентификатор, который используется для обращения к нему из управляющего узла. Узлы могут создавать другие узлы, передавать им команды на выполнение определенных действий, такие как добавление новых данных в хранилище или проверка существующих данных. Также они отвечают на запросы проверки их доступности.

При завершении работы узел отключается от родительского узла и завершает свою работу, освобождая ресурсы.

\section*{Исходный код}

\textbf{socket.hpp}
\lstinputlisting{socket.hpp}

\textbf{socket.cpp}
\lstinputlisting{socket.cpp}

\textbf{topology.hpp}
\lstinputlisting{topology.hpp}

\textbf{topology.cpp}
\lstinputlisting{topology.cpp}

\textbf{calculation\texttt{\_}node.cpp}
\lstinputlisting{calculation_node.cpp}

\textbf{control\texttt{\_}node.cpp}
\lstinputlisting{control_node.cpp}

\textbf{lab5-7\texttt{\_}test.cpp}
\lstinputlisting{lab5-7_test.cpp}

\section*{Демонстрация работы программы}

\textbf{Сама программа}
\begin{verbatim}
arnemkova@LAPTOP-TA2RV74U:~/OS_labs/build$ ./lab5-7/client
create 1 -1
OK: 18007
create 2 1
OK: 18018
create 3 2
OK: 18030
exec 2 may
OK: 2: 'may' not found
exec 2 may 1234
OK: 2
exec 2 may
OK: 2: 1234
ping 1
OK: 1
ping 2
OK: 1
ping 3
OK: 1
kill 1
OK
exit
\end{verbatim}

\textbf{Тесты}
\begin{verbatim}
arnemkova@LAPTOP-TA2RV74U:~/OS_labs/build$ ./tests/lab5-7_test
[==========] Running 2 tests from 1 test suite.
[----------] Global test environment set-up.
[----------] 2 tests from FifthSeventhLabTest
[ RUN      ] FifthSeventhLabTest.SocketTest
[       OK ] FifthSeventhLabTest.SocketTest (1 ms)
[ RUN      ] FifthSeventhLabTest.TopologyTest
[       OK ] FifthSeventhLabTest.TopologyTest (0 ms)
[----------] 2 tests from FifthSeventhLabTest (1 ms total)

[----------] Global test environment tear-down
[==========] 2 tests from 1 test suite ran. (1 ms total)
[  PASSED  ] 2 tests.
\end{verbatim}


\section*{Вывод}

В ходе данной лабораторной работы я познакомилась с библиотекой ZeroMQ, которая позволяет создавать распределенные системы с использованием различных моделей взаимодействия между узлами. Я изучила работу с сокетами ZMQ\texttt{\_}REP и ZMQ\texttt{\_}REQ, которые позволяют установить соединение между управляющим и вычислительным узлами и организовать взаимодействие между ними через запросы и ответы. Также я узнала о принципах мультиплексирования ввода/вывода для эффективной работы с несколькими сокетами одновременно, а также применении асинхронности для параллельной обработки запросов без блокировки основного потока выполнения.

\end{document}
