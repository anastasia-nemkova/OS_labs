\documentclass[a4paper, 14pt]{article}
\usepackage[T1]{fontenc}
\usepackage[utf8]{inputenc}
\usepackage{hyperref}
\hypersetup{
    colorlinks=true,     
    urlcolor=black,
}
\usepackage[english,russian]{babel}
\usepackage{setspace}
\singlespacing
\usepackage{soulutf8}
\usepackage{indentfirst}
\usepackage[top=30mm]{geometry}
\rhead{}
\lhead{}
\renewcommand{\headrulewidth}{0mm}
\usepackage{tocloft}
\usepackage{listings}
\usepackage[dvipsnames]{xcolor}
\usepackage{graphicx}


\lstdefinestyle{mystyle}{
    language=C++,
    commentstyle=\color{YellowGreen},
    keywordstyle=\color{RedViolet},
    numberstyle=\tiny\color{Grey},
    stringstyle=\color{BurntOrange},
    basicstyle=\footnotesize,
    breakatwhitespace=false,         
    breaklines=true,                 
    captionpos=b,                    
    keepspaces=true,                 
    numbers=left,                    
    numbersep=5pt,                  
    showspaces=false,                
    showstringspaces=false,
    showtabs=false,                  
    tabsize=4,
    literate=
        {а}{{\selectfont\char224}}1
        {б}{{\selectfont\char225}}1
        {в}{{\selectfont\char226}}1
        {г}{{\selectfont\char227}}1
        {д}{{\selectfont\char228}}1
        {е}{{\selectfont\char229}}1
        {ё}{{\"e}}1
        {ж}{{\selectfont\char230}}1
        {з}{{\selectfont\char231}}1
        {и}{{\selectfont\char232}}1
        {й}{{\selectfont\char233}}1
        {к}{{\selectfont\char234}}1
        {л}{{\selectfont\char235}}1
        {м}{{\selectfont\char236}}1
        {н}{{\selectfont\char237}}1
        {о}{{\selectfont\char238}}1
        {п}{{\selectfont\char239}}1
        {р}{{\selectfont\char240}}1
        {с}{{\selectfont\char241}}1
        {т}{{\selectfont\char242}}1
        {у}{{\selectfont\char243}}1
        {ф}{{\selectfont\char244}}1
        {х}{{\selectfont\char245}}1
        {ц}{{\selectfont\char246}}1
        {ч}{{\selectfont\char247}}1
        {ш}{{\selectfont\char248}}1
        {щ}{{\selectfont\char249}}1
        {ъ}{{\selectfont\char250}}1
        {ы}{{\selectfont\char251}}1
        {ь}{{\selectfont\char252}}1
        {э}{{\selectfont\char253}}1
        {ю}{{\selectfont\char254}}1
        {я}{{\selectfont\char255}}1
        {А}{{\selectfont\char192}}1
        {Б}{{\selectfont\char193}}1
        {В}{{\selectfont\char194}}1
        {Г}{{\selectfont\char195}}1
        {Д}{{\selectfont\char196}}1
        {Е}{{\selectfont\char197}}1
        {Ё}{{\"E}}1
        {Ж}{{\selectfont\char198}}1
        {З}{{\selectfont\char199}}1
        {И}{{\selectfont\char200}}1
        {Й}{{\selectfont\char201}}1
        {К}{{\selectfont\char202}}1
        {Л}{{\selectfont\char203}}1
        {М}{{\selectfont\char204}}1
        {Н}{{\selectfont\char205}}1
        {О}{{\selectfont\char206}}1
        {П}{{\selectfont\char207}}1
        {Р}{{\selectfont\char208}}1
        {С}{{\selectfont\char209}}1
        {Т}{{\selectfont\char210}}1
        {У}{{\selectfont\char211}}1
        {Ф}{{\selectfont\char212}}1
        {Х}{{\selectfont\char213}}1
        {Ц}{{\selectfont\char214}}1
        {Ч}{{\selectfont\char215}}1
        {Ш}{{\selectfont\char216}}1
        {Щ}{{\selectfont\char217}}1
        {Ъ}{{\selectfont\char218}}1
        {Ы}{{\selectfont\char219}}1
        {Ь}{{\selectfont\char220}}1
        {Э}{{\selectfont\char221}}1
        {Ю}{{\selectfont\char222}}1
        {Я}{{\selectfont\char223}}1
}

\lstset{style=mystyle}

\begin{document}

\thispagestyle{empty}	
\begin{center}
	Московский авиационный институт
	
	(Национальный исследовательский университет)
	
	Факультет информационных технологий и прикладной математики
	
	Кафедра вычислительной математики и программирования
	
\end{center}
\vspace{40ex}
\begin{center}
	\textbf{\large{Лабораторная работа №3 по курсу\linebreak \textquotedblleft Операционные системы\textquotedblright}}
\end{center}
\vspace{35ex}
\begin{flushright}
	\textit{Студент: } Немкова Анастасия Романовна
	
	\vspace{2ex}
	\textit{Группа: } М8О-208Б-22
	
	\vspace{2ex}
	\textit{Преподаватель: } Миронов Евгений Сергеевич
	
	\vspace{2ex}
	\textit{Вариант: 15} 
	
	\vspace{2ex}
	\textit{Оценка: } \underline{\quad\quad\quad\quad\quad\quad}
	
	 \vspace{2ex}
	\textit{Дата: } \underline{\quad\quad\quad\quad\quad\quad}
	
	\vspace{2ex}
	\textit{Подпись: } \underline{\quad\quad\quad\quad\quad\quad}
	
\end{flushright}

\vspace{5ex}

\begin{vfill}
	\begin{center}
		Москва, 2023
	\end{center}	
\end{vfill}
\newpage

\begin{center}
\section*{Содержание}   
\end{center}
\vspace{5ex}
\begin{enumerate}
  \item Репозиторий
  \item Постановка задачи
  \item Общие сведения о программе
  \item Общий метод и алгоритм решения
  \item Исходный код
  \item Демонстрация работы программы
  \item Вывод
\end{enumerate}
\newpage

\begin{figure}
\section*{Репозиторий}   
\vspace{2ex}
\url{https://github.com/anastasia-nemkova/OS_labs}

\section*{Постановка задачи}   
\textbf{Цель работы:}
\vspace{2ex}

Освоение принципов работы с файловыми системами и обеспечение обмена данных между процессами посредством технологии «File mapping»

\vspace{4ex}
\textbf{Задание:}
\vspace{2ex}

    \centering
    \includegraphics[width=0.7\linewidth]{Lab1.png}
    \label{fig:enter-label}
\end{figure}

Составить и отладить программу на языке Си, осуществляющую работу с процессами и взаимодействие между ними в одной из двух операционных систем. В результате работы программа (основной процесс) должен создать для решение задачи один или несколько дочерних процессов. Взаимодействие между процессами осуществляется через системные сигналы/события и/или через отображаемые файлы (memory-mapped files).

Родительский процесс создает дочерний процесс. Первой строкой пользователь в консоль родительского процесса вводит имя файла, которое будет использовано для открытия File с таким именем на запись. Перенаправление стандартных потоков ввода-вывода показано на картинке выше. Родительский и дочерний процесс должны быть представлены разными программами. Родительский процесс принимает от пользователя строки произвольной длины и пересылает их дочернему процессу. Процесс child проверяет строки на валидность правилу. Если строка соответствует правилу, то она выводится в стандартный поток вывода дочернего процесса, иначе в родительский процесс отправляется информация об ошибке. Родительский процесс полученные от child ошибки выводит в стандартный поток вывода\newline

\textit{Вариант 15)} Правило проверки: строка должна начинаться с заглавной буквы

\section*{Общие сведения о программе}

Программа компилируется из файлов parent.cpp с основным процессом, child.cpp с дочерним процессом и utils.cpp c вспомагательными функциями. Также имеются заголовочные файлы parent.hpp и utils.hpp и файл с тестами lab3\texttt{\_}test.cpp. В программе работы были использованы следующие системные вызовы:

\begin{itemize}
    \item sem\texttt{\_}unlink() - удаление именного семафора
    \item shm\texttt{\_}unlink() - удаление именованного сегмента разделяемой памяти 
    \item shm\texttt{\_}open() - создание или открытие объекта разделяемой памяти 
    \item ftruncate() - установка размера разделяемой памяти
    \item mmap() - отображение файлов в адресное пространство процесса 
    \item sem\texttt{\_}open() - создание или открытие именованных семафоров
    \item fork() - создание нового процесса
    \item sem\texttt{\_}post() - увеличение значения(разблокировка) семафора
    \item sem\texttt{\_}wait() - уменьшение значения(блокировка) семафора
    \item sem\texttt{\_}close() - закрытие именованного семафора
    \item munmap() - отключение отображения объекта в адресное пространство процесса
\end{itemize}

\section*{Общий метод и алгоритм решения}

Основной процесс считывает строки из стандартного ввода и записывает их в разделяемую память. После записи процесс уведомляет дочерний процесс о наличии новых данных с помощью семафора. Дочерний процесс, получив уведомление, считывает данные из разделяемой памяти, обрабатывает их, если строки соответствуют правилу - записывает из в файл, иначе записывает ошибочную строку(с маленькой буквы) обратно в разделяемую память, после чего уведомляет основной процесс о готовности результата. После обработки всех строк главный процесс отправляет пустую строку в общую память, чтобы дочерний процесс завершил свою работу. После завершения дочернего процесса главный процесс закрывает общие ресурсы и освобождает память.
\newpage

\section*{Исходный код}

\textbf{parent.hpp}
\lstinputlisting{parent.hpp}

\textbf{parent.cpp}
\lstinputlisting{parent.cpp}

\textbf{utils.hpp}
\lstinputlisting{utils.hpp}

\textbf{utils.cpp}
\lstinputlisting{utils.cpp}

\textbf{child.cpp}
\lstinputlisting{child.cpp}

\textbf{main.cpp}
\lstinputlisting{main.cpp}

\textbf{lab1\texttt{\_}test.cpp}
\lstinputlisting{lab3_test.cpp}
\newpage

\section*{Демонстрация работы программы}
\begin{verbatim}
arnemkova@LAPTOP-TA2RV74U:~/OS_labs/build$ ./tests/lab3_test
[==========] Running 4 tests from 1 test suite.
[----------] Global test environment set-up.
[----------] 4 tests from thirdLabTests
[ RUN      ] thirdLabTests.simpleTest
[       OK ] thirdLabTests.simpleTest (2 ms)
[ RUN      ] thirdLabTests.emptystrTest
[       OK ] thirdLabTests.emptystrTest (1 ms)
[ RUN      ] thirdLabTests.sonneTest
[       OK ] thirdLabTests.sonneTest (2 ms)
[ RUN      ] thirdLabTests.deathTest
[       OK ] thirdLabTests.deathTest (2 ms)
[----------] 4 tests from thirdLabTests (9 ms total)

[----------] Global test environment tear-down
[==========] 4 tests from 1 test suite ran. (9 ms total)
[  PASSED  ] 4 tests.
\end{verbatim}

\section*{Вывод}

В ходе выполнения данной лабораторной работы я изучила принципы работы с файловыми системами и технологию обмена данными между процессами с использованием файловых отображений в память. Я познакомилась с системными вызовами, такими как open, mmap, munmap, shm\texttt{\_}open, ftruncate, sem\texttt{\_}open, sem\texttt{\_}wait, sem\texttt{\_}post. Также изучила работу с семафорами для обеспечения синхронизации доступа к общим ресурсам. При этом я узнала о понятиях таблицы страниц, page hit и page fault, которые играют важную роль при работе с общими сегментами памяти.

\end{document}
