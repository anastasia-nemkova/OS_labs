\documentclass[a4paper, 14pt]{article}
\usepackage[T1]{fontenc}
\usepackage[utf8]{inputenc}
\usepackage{hyperref}
\hypersetup{
    colorlinks=true,     
    urlcolor=black,
}
\usepackage[english,russian]{babel}
\usepackage{setspace}
\singlespacing
\usepackage{soulutf8}
\usepackage{indentfirst}
\usepackage[top=30mm]{geometry}
\rhead{}
\lhead{}
\renewcommand{\headrulewidth}{0mm}
\usepackage{tocloft}
\usepackage{listings}
\usepackage[dvipsnames]{xcolor}
\usepackage{graphicx}
\usepackage{tabularx}
\setlength{\tabcolsep}{8.5pt}
\renewcommand{\arraystretch}{2}
\usepackage{changepage}

\lstdefinestyle{mystyle}{
    language=C++,
    commentstyle=\color{YellowGreen},
    keywordstyle=\color{RedViolet},
    numberstyle=\tiny\color{Grey},
    stringstyle=\color{BurntOrange},
    basicstyle=\footnotesize,
    breakatwhitespace=false,         
    breaklines=true,                 
    captionpos=b,                    
    keepspaces=true,                 
    numbers=left,                    
    numbersep=5pt,                  
    showspaces=false,                
    showstringspaces=false,
    showtabs=false,                  
    tabsize=4,
    literate=
        {а}{{\selectfont\char224}}1
        {б}{{\selectfont\char225}}1
        {в}{{\selectfont\char226}}1
        {г}{{\selectfont\char227}}1
        {д}{{\selectfont\char228}}1
        {е}{{\selectfont\char229}}1
        {ё}{{\"e}}1
        {ж}{{\selectfont\char230}}1
        {з}{{\selectfont\char231}}1
        {и}{{\selectfont\char232}}1
        {й}{{\selectfont\char233}}1
        {к}{{\selectfont\char234}}1
        {л}{{\selectfont\char235}}1
        {м}{{\selectfont\char236}}1
        {н}{{\selectfont\char237}}1
        {о}{{\selectfont\char238}}1
        {п}{{\selectfont\char239}}1
        {р}{{\selectfont\char240}}1
        {с}{{\selectfont\char241}}1
        {т}{{\selectfont\char242}}1
        {у}{{\selectfont\char243}}1
        {ф}{{\selectfont\char244}}1
        {х}{{\selectfont\char245}}1
        {ц}{{\selectfont\char246}}1
        {ч}{{\selectfont\char247}}1
        {ш}{{\selectfont\char248}}1
        {щ}{{\selectfont\char249}}1
        {ъ}{{\selectfont\char250}}1
        {ы}{{\selectfont\char251}}1
        {ь}{{\selectfont\char252}}1
        {э}{{\selectfont\char253}}1
        {ю}{{\selectfont\char254}}1
        {я}{{\selectfont\char255}}1
        {А}{{\selectfont\char192}}1
        {Б}{{\selectfont\char193}}1
        {В}{{\selectfont\char194}}1
        {Г}{{\selectfont\char195}}1
        {Д}{{\selectfont\char196}}1
        {Е}{{\selectfont\char197}}1
        {Ё}{{\"E}}1
        {Ж}{{\selectfont\char198}}1
        {З}{{\selectfont\char199}}1
        {И}{{\selectfont\char200}}1
        {Й}{{\selectfont\char201}}1
        {К}{{\selectfont\char202}}1
        {Л}{{\selectfont\char203}}1
        {М}{{\selectfont\char204}}1
        {Н}{{\selectfont\char205}}1
        {О}{{\selectfont\char206}}1
        {П}{{\selectfont\char207}}1
        {Р}{{\selectfont\char208}}1
        {С}{{\selectfont\char209}}1
        {Т}{{\selectfont\char210}}1
        {У}{{\selectfont\char211}}1
        {Ф}{{\selectfont\char212}}1
        {Х}{{\selectfont\char213}}1
        {Ц}{{\selectfont\char214}}1
        {Ч}{{\selectfont\char215}}1
        {Ш}{{\selectfont\char216}}1
        {Щ}{{\selectfont\char217}}1
        {Ъ}{{\selectfont\char218}}1
        {Ы}{{\selectfont\char219}}1
        {Ь}{{\selectfont\char220}}1
        {Э}{{\selectfont\char221}}1
        {Ю}{{\selectfont\char222}}1
        {Я}{{\selectfont\char223}}1
}

\lstset{style=mystyle}

\begin{document}

\thispagestyle{empty}	
\begin{center}
	Московский авиационный институт
	
	(Национальный исследовательский университет)
	
	Факультет информационных технологий и прикладной математики
	
	Кафедра вычислительной математики и программирования
	
\end{center}
\vspace{40ex}
\begin{center}
	\textbf{\large{Лабораторная работа №4 по курсу\linebreak \textquotedblleft Операционные системы\textquotedblright}}
\end{center}
\vspace{35ex}
\begin{flushright}
	\textit{Студент: } Немкова Анастасия Романовна
	
	\vspace{2ex}
	\textit{Группа: } М8О-208Б-22
	
	\vspace{2ex}
	\textit{Преподаватель: } Миронов Евгений Сергеевич
	
	\vspace{2ex}
	\textit{Вариант: 13} 
	
	\vspace{2ex}
	\textit{Оценка: } \underline{\quad\quad\quad\quad\quad\quad}
	
	 \vspace{2ex}
	\textit{Дата: } \underline{\quad\quad\quad\quad\quad\quad}
	
	\vspace{2ex}
	\textit{Подпись: } \underline{\quad\quad\quad\quad\quad\quad}
	
\end{flushright}

\vspace{5ex}

\begin{vfill}
	\begin{center}
		Москва, 2023
	\end{center}	
\end{vfill}
\newpage

\begin{center}
\section*{Содержание}   
\end{center}
\vspace{5ex}
\begin{enumerate}
  \item Репозиторий
  \item Постановка задачи
  \item Общие сведения о программе
  \item Общий метод и алгоритм решения
  \item Исходный код
  \item Демонстрация работы программы
  \item Вывод
\end{enumerate}
\newpage


\section*{Репозиторий}   
\vspace{2ex}
\url{https://github.com/anastasia-nemkova/OS_labs}

\section*{Постановка задачи}   
\textbf{Цель работы:}
\vspace{2ex}

Создание динамических библиотек и создание программ, которые используют функции динамических библиотек

\vspace{4ex}
\textbf{Задание:}
\vspace{2ex}

Требуется создать динамические библиотеки, которые реализуют определенный функционал. 
Далее использовать данные библиотеки 2-мя способами:
\begin{enumerate}
    \item Во время компиляции (на этапе «линковки»/linking)
    \item Во время исполнения программы. Библиотеки загружаются в память с помощью интерфейса ОС для работы с динамическими библиотеками
\end{enumerate}
В конечном итоге, в лабораторной работе необходимо получить следующие части:
\begin{itemize}
    \item Динамические библиотеки, реализующие контракты, которые заданы вариантом;
    \item Тестовая программа (программа №1), которая используют одну из библиотек, используя знания полученные на этапе компиляции;
    \item Тестовая программа (программа №2), которая загружает библиотеки, используя только их местоположение и контракты.
\end{itemize}

Пользовательский ввод для обоих программ должен быть организован следующим образом:
\begin{enumerate}
    \item Если пользователь вводит команду «0», то программа переключает одну реализацию контрактов на другую (необходимо только для программы №2).
    \item «1 arg1 arg2 … argN», где после «1» идут аргументы для первой функции, предусмотренной контрактами. После ввода команды происходит вызов первой функции, и на экране появляется результат её выполнения;
    \item «2 arg1 arg2 … argM», где после «2» идут аргументы для второй функции, предусмотренной контрактами. После ввода команды происходит вызов второй функции, и на экране появляется результат её выполнения.
\end{enumerate}
\newpage
\textit{Вариант 13:}
\begin{table}[h]
\caption{Контракты и реализации функций}
\medskip
\begin{adjustwidth}{-1,5cm}{0cm}
\medskip
\begin{tabular}{ |>{\raggedright}p{0.1cm}|>{\raggedright}p{3cm}|>{\raggedright}p{2,5cm}|>{\raggedright}p{4cm}|>{\raggedright\arraybackslash}p{5cm}|  }
\hline
№ & Описание & Сигнатура & Реализация 1 & Реализация 2 \\
\hline
2 & Рассчет производной функции $\cos(x)$ в точке $A$ с приращением $\delta x$ & \texttt{Float Derivative (float A, float deltaX)} & $f'(x) =$ $ \frac{f(A+\text{deltaX}) - f(A)}{\text{deltaX}}$ & $f'(x) = \frac{f(A+\text{deltaX}) - f(A-\text{deltaX})}{2\cdot\text{deltaX}}$ \\
\hline
7 & Подсчет площади плоской геометрической фигуры по двум сторонам & \texttt{Float Square(float A, float B)} & Фигура прямоугольник & Фигура прямоугольный треугольник \\
\hline
\end{tabular}
\end{adjustwidth}
\end{table}

\section*{Общие сведения о программе}
Программа компилируется из файлов realizations.hpp, realization1.cpp, realization2.cpp,
static\texttt{\_}main.cpp, dynamic\texttt{\_}main.cpp. Также имеются файлы с тестами lab4\texttt{\_}realiz1\texttt{\_}test.cpp и lab4\texttt{\_}realiz2\texttt{\_}test.cpp  В программе работы были использованы следующие системные вызовы:

\begin{itemize}
    \item dlsym() - получение адреса функции из динамической библиотеки
    \item dlopen() - открытие динамической библиотеки
    \item dlclose() - закрытие динамеческой библиотеки
    \item dlerror() - возвращение строки, описывающей последнюю ошибку, произошедшую при вызове функций из динамической библиотеки
\end{itemize}

\section*{Общий метод и алгоритм решения}

У нас имеется статическая и динамическая реализации. Для первой из них мы компилируем основной файл вместе с динамической библиотекой. Для второй используем переменные окружения PATH\texttt{\_}TO\texttt{\_}LIB1 и PATH\texttt{\_}TO\texttt{\_}LIB2 для определения путей к двум различным динамическим библиотекам. 

В динамической реализации основной файл программы загружает библиотеки с помощью системного вызова dlopen и получает указатели на функции из этих библиотек с помощью dlsym. При вводе команды "0" пользователем программа переключает одну библиотеку на другую.

В статической реализации функции из динамических библиотек вызываются напрямую из основного файла программы, так как эти функции уже были статически связаны с основным файлом в процессе компиляции
\newpage

\section*{Исходный код}

\textbf{realizations.hpp}
\lstinputlisting{realizations.hpp}

\textbf{realization1.cpp}
\lstinputlisting{realization1.cpp}

\textbf{realization2.cpp}
\lstinputlisting{realization2.cpp}

\textbf{static\texttt{\_}main.cpp}
\lstinputlisting{static_main.cpp}

\textbf{dynamic\texttt{\_}main.cpp}
\lstinputlisting{dynamic_main.cpp}

\textbf{lab4\texttt{\_}realiz1\texttt{\_}test.cpp}
\lstinputlisting{lab4_realiz1_test.cpp}

\textbf{lab4\texttt{\_}realiz2\texttt{\_}test.cpp}
\lstinputlisting{lab4_realiz2_test.cpp}


\section*{Демонстрация работы программы}

\textbf{Тесты для статической реализации}

\begin{verbatim}
arnemkova@LAPTOP-TA2RV74U:~/OS_labs/build$ ./lab4/static_main
Enter command (0 to exit): 1 3 5
Result of Derivative: 0.168898
Enter command (0 to exit): 2 4 5
Result of Square: 20
Enter command (0 to exit): 0
\end{verbatim}

\textbf{Тесты для динамической реализации}

\begin{verbatim}
arnemkova@LAPTOP-TA2RV74U:~/OS_labs/build$ ./lab4/dynamic_main
Enter command (0 to switch libraries, 1 or 2 to call functions, q to exit): 1
 3 5
Result of Derivative: 0.168898
Enter command (0 to switch libraries, 1 or 2 to call functions, q to exit): 2 4 5
Result of Square: 20
Enter command (0 to switch libraries, 1 or 2 to call functions, q to exit): 0
Enter library number (1 or 2): 2
Enter command (0 to switch libraries, 1 or 2 to call functions, q to exit): 1 3 5
Result of Derivative: 0.0844492
Enter command (0 to switch libraries, 1 or 2 to call functions, q to exit): 2 4 5
Result of Square: 10
\end{verbatim}

\textbf{Тесты}

\begin{verbatim}
arnemkova@LAPTOP-TA2RV74U:~/OS_labs/build$ ./tests/lab4_realiz1_test
[==========] Running 2 tests from 1 test suite.
[----------] Global test environment set-up.
[----------] 2 tests from FourthLabTest
[ RUN      ] FourthLabTest.DerivativeStaticTest
[       OK ] FourthLabTest.DerivativeStaticTest (0 ms)
[ RUN      ] FourthLabTest.SquareStaticTest
[       OK ] FourthLabTest.SquareStaticTest (0 ms)
[----------] 2 tests from FourthLabTest (0 ms total)

[----------] Global test environment tear-down
[==========] 2 tests from 1 test suite ran. (0 ms total)
[  PASSED  ] 2 tests.
arnemkova@LAPTOP-TA2RV74U:~/OS_labs/build$ ./tests/lab4_realiz2_test
[==========] Running 2 tests from 1 test suite.
[----------] Global test environment set-up.
[----------] 2 tests from FourthLabTest
[ RUN      ] FourthLabTest.DerivativeStaticTest
[       OK ] FourthLabTest.DerivativeStaticTest (0 ms)
[ RUN      ] FourthLabTest.SquareStaticTest
[       OK ] FourthLabTest.SquareStaticTest (0 ms)
[----------] 2 tests from FourthLabTest (0 ms total)

[----------] Global test environment tear-down
[==========] 2 tests from 1 test suite ran. (0 ms total)
[  PASSED  ] 2 tests.
\end{verbatim}

\section*{Вывод}

В ходе данной лабораторной работы я познакомилась с использованием динамических библиотек в ОС Linux, которые позволяют программе загружать и использовать функции из библиотек во время выполнения, что обеспечивает гибкость и возможность изменения программы без перекомпиляции. Также я узнала про этапы сборки программы и особенности использования extern "C" при линковке файлов с общим include.  

\end{document}
